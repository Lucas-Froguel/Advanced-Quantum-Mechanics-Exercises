\documentclass[12pt]{article}
\usepackage{amsfonts}
\usepackage{amsthm}
\usepackage{amsmath}
\usepackage[mathscr]{euscript}
\usepackage{array}
\usepackage[thinlines]{easytable}
\usepackage{tikz}
\usepackage{pgfplots}
\usepackage[margin=2cm]{geometry}
\usepackage{graphicx}
\usepackage{xcolor}
\usepackage[utf8]{inputenc}
\usepackage[T1]{fontenc}
\usepackage[portuguese]{babel}
\usepackage{braket}
\usepackage{thmtools} 
\usepackage{hyperref}
\usepackage{environ}
\usepackage{enumitem}
\usepackage[backend=biber, style=numeric]{biblatex} %Imports biblatex package
\addbibresource{references.bib} %Import the bibliography file
%
\usepackage{lipsum} % Pode tirar esse :D

\hypersetup{
	colorlinks=true,
	linkcolor=blue,
	filecolor=magenta,      
	urlcolor=cyan,
	pdftitle={Quântica Avançada - L1 - Lucas Froguel}
}


\def\be{\begin{equation}}
	\def\ee{\end{equation}}
\def\bee{\begin{equation*}}
	\def\eee{\end{equation*}}
\def\bea{\begin{eqnarray*}}
	\def\eea{\end{eqnarray*}}
\def\beaa{\begin{eqnarray}}
	\def\eeaa{\end{eqnarray}}

\def\f{\frac}
\def\del{\partial}

\def\R{\mathbb{R}}
\def\K{\mathbb{K}}
\def\C{\mathbb{C}}
\def\I{\mathbb{I}}
\def\Z{\mathbb{Z}}
\def\Q{\mathbb{Q}}
\def\N{\mathbb{N}}

\def\cd{\cdot}

\def\v#1{{\boldsymbol{#1}}}
\def\ve#1{\hat{\boldsymbol#1}}

\def\l{\left}
\def\r{\right}
\def\la{\l\langle}
\def\ra{\r\rangle}
\def\div{\nabla\cdot}
\def\curl{\nabla\times}
\def\grad{\nabla}
\def\lap{\nabla^2}

\def\s{\quad}
\def\ss{\qquad}
\def\infi{\infty}
\def\p{\partial}
\def\u{\cup}%union of two sets
\def\i{\cap}%intersection of two sets
\def\ds{\oplus}



\newtheorem{exercise}{Exercise}
\newtheorem{partinner}{Part}[exercise]

\newlist{exercises}{enumerate}{1}
\setlist[exercises]{wide = 0pt, listparindent=\parindent,labelsep = 0pt,leftmargin =\labelwidth}
\setlist[exercises, 1]{label =\itshape \bfseries Part~\arabic*.\; }

\newenvironment{answer}{\noindent\textbf{\textit{Answer.}} \normalfont }{\par\noindent\rule{\textwidth}{0.4pt}}
%\newenvironment{multianswer}{\\ \noindent\textbf{\textit{Answer.}} \normalfont }{ \par\noindent\rule{0.7\textwidth}{0.1pt}}
\NewEnviron{multianswer}[1][false]{%
	\ifthenelse{\equal{#1}{true}}%
	{\def\rulewidth{\textwidth}}%
	{\def\rulewidth{0.7\textwidth}}%
	\\ \noindent\textbf{\textit{Answer.}} \normalfont%
	\BODY%
	\par\noindent\rule{\rulewidth}{0.1pt}%
}


\newcommand\norm[1]{\left\lVert#1\right\rVert}

\DeclareMathOperator{\atan}{atan}
\DeclareMathOperator{\acos}{acos}
\DeclareMathOperator{\sech}{sech}
\DeclareMathOperator{\csch}{csch}
\DeclareMathOperator{\asinh}{asinh}
\DeclareMathOperator{\atanh}{atanh}
\DeclareMathOperator{\acoth}{acoth}
\DeclareMathOperator{\acosh}{acosh}
\DeclareMathOperator{\acsch}{acsch}
\DeclareMathOperator{\asech}{asech}
\DeclareMathOperator{\D}{D}
\DeclareMathOperator{\Tr}{tr}

\title{Mecânica Quântica Avançada \\ Lista 1}
\author{Lucas Froguel \\ IFT}
\date{}

\begin{document}
	\maketitle
	\listoftheorems[title={List of Exercises}]

	\begin{exercise}[4.1]
		Show that a necessary and sufficient condition for $|\psi\rangle$ to be an eigenvector of a Hermitian
		operator A is that the dispersion (4.8) $\Delta_\psi A = 0$.
	\end{exercise}
	\begin{answer}
		Vamos iniciar mostrando que se $\ket{\psi}=0$, então $\Delta_\psi A = 0$. Ora, por definição:
		\be
			\Delta_\psi A = \braket{\psi|A^2|\psi} - \braket{\psi|A|\psi}^2
		\ee
		É fácil ver que
		\be
			\braket{\psi|A^2|\psi} = \bra{\psi|A^\dagger}\ket{A|\psi} = a^2
		\ee
		E também
		\be
			\braket{\psi|A|\psi}^2 = \l(a\r)^2 = a^2
		\ee
		Logo, é trivial que $\Delta_\psi A = 0$. 
		
		Agora vamos assumir que a dispersão é nula, ou seja, $\Delta A = 0$. Então ,por definição:		
		\begin{align}
			0 &= \sqrt{\langle A^2 \rangle - \langle A \rangle^2} \\
			\Rightarrow \langle A^2 \rangle &= \langle A \rangle^2
		\end{align}		
				
	\end{answer}

	\begin{exercise}[4.4.2 - 1]
		 Let $\ket{\psi}$ be a vector (not normalized) in the Hilbert space of states and H be a Hamiltonian. The
		expectation value $\braket{H}_\psi$ is
		\be
			\braket{\psi}_\psi = \f{\braket{\psi|H|\psi}}{\braket{\psi|\psi}}
		\ee
		Show that if the minimum of this expectation value is obtained for $\ket{\psi}=\ket{\psi_m}$ and the maximum for $\ket{\psi}=\ket{\psi_M}$ , then
		\be
			H\ket{\psi_m} = E_m\ket{\psi_m}, \quad\quad H\ket{\psi_M} = E_M\ket{\psi_M}
		\ee 	
		where $E_m$ and $E_M$ are the smallest and largest eigenvalues.
	\end{exercise}
	\begin{answer}
		É evidente que
		\be
			\braket{H}_{\psi_m} = \f{\braket{\psi_m|H|\psi_m}}{\braket{\psi_m|\psi_m}} = E_m
		\ee
		Portanto, é evidente que se $\braket{H}_{\psi_m}$ for mínimo, então $E_m$ também é. Vale um raciocínio análogo para $E_M=\braket{H}_{\psi_M}$. 		
	\end{answer}
	
	\begin{exercise}[4.4.2 - 2]
		We assume that the vector $|\varphi\rangle$ depends on a parameter $\alpha:|\varphi\rangle=|\varphi(\alpha)\rangle$. Show that if
		\begin{equation}
			\frac{\partial\braket{H}_{\varphi(\alpha)}}{\partial \alpha}\bigg|_{\alpha=\alpha_0}=0,
		\end{equation}
		then $E_{m} \leq\langle H\rangle_{\varphi\left(\alpha_{0}\right)}$ if $\alpha_{0}$ corresponds to a minimum of $\langle H\rangle_{\varphi(\alpha)}$, and $\langle H\rangle_{\varphi\left(\alpha_{0}\right)} \leq E_{M}$ if $\alpha_{0}$ corresponds to a maximum. This result forms the basis of an approximation method called the variational method (Section 14.1.4).
	\end{exercise}
	\begin{answer}
		Vamos abrir a derivada:
		\be
			\frac{\partial\braket{H}{\varphi(\alpha)}}{\partial \alpha} = \f{1}{\braket{\psi|\psi}}\l( \bra{\psi|H} \partial_\alpha\ket{\psi} + (\partial_\alpha\bra{\psi})\ket{H|\psi}\r) - \f{\braket{\psi|H|\psi}}{\braket{\psi|\psi}^2}\l( (\partial_\alpha\bra{\psi})\ket{\psi} + \bra{\psi}\partial_\alpha\ket{\psi}\r) 
		\ee
		Então, em $\alpha_0$
		\be
			\braket{\psi|H|\partial_\alpha\psi} + \braket{\partial_\alpha\psi|H|\psi} = \f{\braket{\psi|H|\psi}}{\braket{\psi|\psi}}\big( \braket{\partial_\alpha\psi|\psi} + \braket{\psi|\partial_\alpha\psi}\big)
		\ee
		Isolando a quantidade de interesse:
		\be
			\braket{H}_{\varphi(\alpha_0)} = \f{\braket{\psi|H|\partial_\alpha\psi} + \braket{\partial_\alpha\psi|H|\psi}}{\braket{\partial_\alpha\psi|\psi} + \braket{\psi|\partial_\alpha\psi}}
		\ee 
		Podemos reescrever, usando que $H=H^\dagger$:
		\be
			\braket{H}_{\varphi(\alpha_0)} = \f{\braket{\partial_\alpha\psi|H|\psi}^\dagger + \braket{\partial_\alpha\psi|H|\psi}}{\braket{\partial_\alpha\psi|\psi}^\dagger + \braket{\partial_\alpha\psi|\psi}}
		\ee
		
		Podemos expandir qualquer estado usando os autokets do hamiltoniano, que são ortonormais:
		\be
			\ket{\psi} = \sum c_j\ket{\psi_j}
		\ee
		De modo que
		\be
			\partial_\alpha\ket{\psi} = \sum \f{\partial c_j}{\partial\alpha} \ket{\psi_j}
		\ee
		Assim, vale que
		\be
			\braket{\partial_\alpha\psi|\psi} = \sum c_j\f{\partial c_j^*}{\partial\alpha}
		\ee
		E também
		\be
			\braket{\partial_\alpha\psi|H|\psi} = \sum E_jc_j\f{\partial c_j^*}{\partial\alpha}
		\ee
		Considere o denominador:
		\bea
			\braket{\partial_\alpha\psi|\psi}^\dagger + \braket{\partial_\alpha\psi|\psi} 
				&=& \sum c_j \f{\partial c_j^*}{\partial\alpha} + c_j^*\f{\partial c_j}{\partial\alpha} \\
				&=& \sum \f{\partial}{\partial\alpha} (c_j^*c_j) \\
				&=& \sum \partial_\alpha |c_j|^2 \\
				&=& \partial_\alpha \sum |c_j|^2
		\eea
		Considerando, enfim, o numerador e fazendo os mesmos cálculos:
		\bea
			\braket{\partial_\alpha\psi|H|\psi}^\dagger + \braket{\partial_\alpha\psi|H|\psi} 
				&=& \sum E_jc_j\f{\partial c_j^*}{\partial\alpha} + E_j c_* \f{\partial c_j}{\partial\alpha} \\
				&=& \sum E_j \partial_\alpha |c_j|^2 \\
				&=& \partial_\alpha \sum E_j|c_j|^2 
		\eea
		É óbvio, então, que
		\be
			E_m \partial_\alpha\sum |c_j|^2 \leq \partial_\alpha\sum E_j|c_j|^2 \leq E_M \partial_\alpha\sum |c_j|^2
		\ee 
		Portanto, concluímos que
		\be
			E_m \leq \braket{H}_{\varphi(\alpha_0)} \leq E_M
		\ee
		
	\end{answer}

	\begin{exercise}[4.4.2 - 3]
		If $H$ acts in a two-dimensional space, its most general form is
		$$
		H=\left(\begin{array}{cc}
			a+c & b \\
			b & a-c
		\end{array}\right),
		$$
		where $b$ can always be chosen to be real. Parametrizing $|\varphi(\alpha)\rangle$ as
		
		$$
		|\varphi(\alpha)\rangle=\left(\begin{array}{c}
			\cos \alpha / 2 \\
			\sin \alpha / 2
		\end{array}\right)
		$$
		find the values of $\alpha_{0}$ by seeking the extrema of $\langle\varphi(\alpha)|H| \varphi(\alpha)\rangle$. Rederive (2.35).
	\end{exercise}
	\begin{answer}
	Começamos considerando a equação do valor médio explicitamente:
	\bea
		\braket{\psi|H|\psi} 
			&=& 
			\begin{bmatrix}
				\cos(\alpha/2) & \sin(\alpha/2)
			\end{bmatrix}
			\begin{bmatrix}
				a+c & b \\
				b & a-c
			\end{bmatrix}
			\begin{bmatrix}
				\cos(\alpha/2) \\
				\sin(\alpha/2)
			\end{bmatrix} \\
			&=& 
			\begin{bmatrix}
				\cos(\alpha/2) & \sin(\alpha/2)
			\end{bmatrix}
			\begin{bmatrix}
				(a+c)\cos(\alpha/2) + b\sin(\alpha/2) \\
				b\cos(\alpha/2) + (a-c)\sin(\alpha/2)
			\end{bmatrix} \\
			&=& 
			(a+c)\cos^2(\alpha/2) + b\sin(\alpha/2)\cos(\alpha/2) + b\cos(\alpha/2)\sin(\alpha/2) + (a-c)\sin^2(\alpha/2) \\
			&=& 
			a(\sin^2(\alpha/2)+ \cos^2(\alpha/2)) + c(\cos^2(\alpha/2) - \sin^2(\alpha/2)) + b\sin(\alpha) \\
			&=& a + b\sin(\alpha) + c\cos(\alpha)
	\eea
	Vejamos os extremos dessa função:
	\bea	
		\partial_\alpha \braket{\psi|H|\psi} &=& b\cos(\alpha) - c\sin(\alpha) = 0 \\
		b\cos(\alpha) &=& c\sin(\alpha) \\
		\tan(\alpha) &=& b/c \\
		\alpha_0 &=& \atan(b/c)
	\eea
	A Eq.(2.35) se refere aos autovetores e autovalores de $H$. Vejamos o caso do nosso $\psi$:
	\bea	
		H\ket{\psi} &=& 
		\l( a\begin{bmatrix}
			  1 & 0 \\
			  0 & 1
			\end{bmatrix}
			+ \sqrt{b^2+c^2}
			\begin{bmatrix}
				\cos\alpha & \sin\alpha \\
				\sin\alpha & -\cos\alpha
			\end{bmatrix}
		\r)
		\begin{bmatrix}
			\cos(\alpha/2) \\
			\sin(\alpha/2)
		\end{bmatrix} \\
		&=& 
		a\begin{bmatrix}
			\cos(\alpha/2) \\
			\sin(\alpha/2)
		\end{bmatrix} 
		+ 
		\sqrt{b^2+c^2}
		\begin{bmatrix}
			\cos\alpha\cos(\alpha/2) + \sin\alpha\sin(\alpha/2) \\
			\sin\alpha\cos(\alpha/2) - \cos\alpha\sin(\alpha/2)
		\end{bmatrix} \\
		&=& 
		a + \sqrt{b^2+c^2} 
		\begin{bmatrix}
			\cos(\alpha/2) \\
			\sin(\alpha/2)
		\end{bmatrix} 
	\eea
	onde usamos a Eq.(2.34) e algumas identidades trigonométricas.	
	\end{answer}
	
	\begin{exercise}[4.4.3]
		Let a Hamiltonian $H$ depend on a parameter $\lambda: H=H(\lambda)$. Let $E(\lambda)$ be a nondegenerate eigenvalue and $|\varphi(\lambda)\rangle$ be the corresponding normalized eigenvector $\left(\|\varphi(\lambda)\|^{2}=1\right)$ :
		
		$$
		H(\lambda)|\varphi(\lambda)\rangle=E(\lambda)|\varphi(\lambda)\rangle
		$$
		Demonstrate the Feynman-Hellmann theorem:
		$$
		\frac{\partial E}{\partial \lambda}=\left\langle\varphi(\lambda)\left|\frac{\partial H}{\partial \lambda}\right| \varphi(\lambda)\right\rangle .
		$$
	\end{exercise}
	\begin{answer}
		Sabemos que podemos escrever
		\be
			E(\lambda) = \braket{\psi|H|\psi}
		\ee
		Então considere:
		\be
			\partial_\lambda E = \braket{\partial_\lambda\psi|H|\psi} + \braket{\psi|\partial_\lambda|\psi} + \braket{\psi|H|\partial_\lambda\psi} 
		\ee
		Logo, é suficiente mostrar que 
		\be
			\braket{\partial_\lambda\psi|H|\psi} +  \braket{\psi|H|\partial_\lambda\psi} = 0
		\ee
		Considere:
		\bea
			\braket{\partial_\lambda\psi|H|\psi} +  \braket{\psi|H|\partial_\lambda\psi} 
			&=& \braket{\partial_\lambda\psi|E|\psi} + \braket{\psi|E^\dagger|\partial_\lambda\partial_\lambda\psi} \\
 			&=& E \l(\braket{\partial_\lambda\psi|\psi} + \braket{\psi|\partial_\lambda\psi}\r) \\ 
			&=& E( \partial_\lambda\braket{\psi|\psi}) \\
			&=& E\partial_\lambda 1 \\
			&=& 0
		\eea
		onde usamos o fato de $H$ ser hermitiano e ter autovalores reais e de $\ket{\psi}$ ser normalizado. Portanto, fica demonstrado o teorema.		
	\end{answer}

	\begin{exercise}[4.4.4]
		We consider a two-level system with Hamiltonian $H$ represented by the matrix
		$$
		H=\hbar\left(\begin{array}{cc}
			A & B \\
			B & -A
		\end{array}\right)
		$$
		in the basis
		$$
		|+\rangle=\left(\begin{array}{l}
			1 \\
			0
		\end{array}\right), \quad|-\rangle=\left(\begin{array}{l}
			0 \\
			1
		\end{array}\right)
		$$
		According to (2.35), the eigenvalues and eigenvectors of $H$ are
		$$
		\begin{array}{ll}
			E_{+}=\hbar \sqrt{A^{2}+B^{2}}, & \left|\chi_{+}\right\rangle=\cos \frac{\theta}{2}|+\rangle+\sin \frac{\theta}{2}|-\rangle \\
			E_{-}=-\hbar \sqrt{A^{2}+B^{2}}, & \left|\chi_{-}\right\rangle=-\sin \frac{\theta}{2}|+\rangle+\cos \frac{\theta}{2}|-\rangle
		\end{array}
		$$
		with
		$$
		A=\sqrt{A^{2}+B^{2}} \cos \theta, \quad B=\sqrt{A^{2}+B^{2}} \sin \theta, \quad \tan \theta=\frac{B}{A}
		$$
		\begin{exercises}
		\item The state vector $|\varphi(t)\rangle$ at time $t$ can be decomposed on the $\{|+\rangle,|-\rangle\}$ basis:
		
		$$
		|\varphi(t)\rangle=c_{+}(t)|+\rangle+c_{-}(t)|-\rangle
		$$
		
		Write down the system of coupled differential equations which the components $c_{+}(t)$ and $c_{-}(t)$ satisfy.
		\begin{multianswer}
				Começamos escrevendo:
				\be
					\ket{\varphi(t)} = \begin{pmatrix}
						c_+(t) \\ c_-(t)
						\end{pmatrix}
				\ee
				Agora montamos a equação de Schrödinger:
				\bea
					H\ket{\varphi(t)} &=& -i\hbar \partial_t\ket{\varphi(t)} \\
					\hbar\begin{pmatrix}
						A & B \\
						B & -A
					\end{pmatrix} 
					\begin{pmatrix}
						c_+(t) \\ c_-(t)
					\end{pmatrix} 
					&=& 
					-i\hbar 
					\begin{pmatrix}
						\partial_t c_+(t) \\ \partial_t c_-(t)
					\end{pmatrix} \\
					\begin{pmatrix}
						Ac_+ + Bc_- \\ Bc_+ - Ac_-
					\end{pmatrix}
					&=&
					-i \begin{pmatrix}
						\dot{c}_+(t) \\ \dot{c}_-(t)
					\end{pmatrix}
				\eea
			Ou seja, temos as equações diferenciais:
			\beaa
				i\dot{c}_+ &=& Ac_+ + Bc_- \\
				i\dot{c}_- &=& Bc_+ - Ac_-
			\eeaa				
		\end{multianswer}
		
		\item Let $|\varphi(t=0)\rangle$ be decomposed on the $\left\{\left|\chi_{+}\right\rangle,\left|\chi_{-}\right\rangle\right\}$basis:
		$$
		|\varphi(t=0)\rangle=|\varphi(0)\rangle=\lambda\left|\chi_{+}\right\rangle+\mu\left|\chi_{-}\right\rangle, \quad|\lambda|^{2}+|\mu|^{2}=1
		$$
		Show that $c_{+}(t)=\langle+\mid \varphi(t)\rangle$ is written as
		$$
		c_{+}(t)=\lambda \mathrm{e}^{-\mathrm{i} \Omega t / 2} \cos \frac{\theta}{2}-\mu \mathrm{e}^{\mathrm{i} \Omega t / 2} \sin \frac{\theta}{2}
		$$
		with $\Omega=2 \sqrt{A^{2}+B^{2}}$. Here $\hbar \Omega$ is the energy difference of the two levels. Show that $c_{+}(t)$ (as well as $\left.c_{-}(t)\right)$ satisfies the differential equation
		$$
		\ddot{c}_{+}(t)+\left(\frac{\Omega}{2}\right)^{2} c_{+}(t)=0 .
		$$
		\begin{multianswer}
			Vamos começar pela segunda parte, mostrando a validez da equação diferencial. Considere as EDOs que obtemos e vamos deriva-las mais uma vez (faremos as contas para $c_+$, pois são análogas para $c_-$.):
			\be
				i\ddot{c}_+ = A\dot{c}_+ + B\dot{c}_-
			\ee
			Mas note, também, que:
			\bea
				A\dot{c}_+ &=& -iA^2c_+ - iABc_- \\
				B\dot{c}_- &=& -iB^2c_+ + iABc_-
			\eea
			Logo, 
			\be
				A\dot{c}_+ + B\dot{c}_- = -i(A^2+B^2)c_+
			\ee
			Portanto,
			\be
				i\ddot{c}_+ = -i(A^2+B^2)c_+
			\ee
			que, simplificando, implica na expressão desejada
			\be
				\ddot{c}_+ +\l(\f{\Omega}{2}\r)^2c_+ = 0
			\ee
			Vamos considerar agora a outra parte. Ela nos dá condições iniciais:
			\be
				\ket{\varphi_0} = 
				\begin{pmatrix}
					\lambda\cos(\theta/2) - \mu\sin(\theta/2) \\
					\lambda\sin(\theta/2) + \mu\cos(\theta/2) \\
				\end{pmatrix}
			\ee
			Ou seja,
			\bea
				c_+(0) &=& \lambda\cos(\theta/2) - \mu\sin(\theta/2) \\
				c_-(0) &=& \lambda\sin(\theta/2) + \mu\cos(\theta/2) 
			\eea
			Usando a equação diferencial que deduzimos, é fácil ver que:
			\bea
				c_+(t) &=& A_+e^{i\Omega t/2} + B_+e^{-i\Omega t/2} \\
				c_-(t) &=& A_-e^{i\Omega t/2} + B_-e^{-i\Omega t/2} 
			\eea
			Isso obviamente nos diz que:
			\bea
				A_+ + B_+ &=& \lambda\cos(\theta/2) - \mu\sin(\theta/2) \\
				A_- + B_- &=& \lambda\sin(\theta/2) + \mu\cos(\theta/2) 
			\eea
			*
		\end{multianswer}
		\item We assume that $c_{+}(0)=0$. Find $\lambda$ and $\mu$ up to a phase as well as $c_{+}(t)$. Show that the probability of finding the system in the state $|+\rangle$ at time $t$ is
		
		$$
		\mathrm{p}_{+}(t)=\sin ^{2} \theta \sin ^{2}\left(\frac{\Omega t}{2}\right)=\frac{B^{2}}{A^{2}+B^{2}} \sin ^{2}\left(\frac{\Omega t}{2}\right) .
		$$
		\begin{multianswer}[true]
			Se $c_+(0)=0$, então precisa ser que
			\be
				 \lambda\cos(\theta/2) - \mu\sin(\theta/2) = 0
			\ee
			Logo, é verdade que
			\be
				\lambda = \mu \tan(\theta/2)
			\ee
			Mas, pela normalização:
			\bea
				|\lambda|^2 + |\mu|^2 &=& 1 \\
				|\mu\tan(\theta/2)|^2 + |\mu|^2 &=& 1 \\
				|\mu|^2|\tan(\theta/2)|^2 + |\mu|^2 &=& \\
				|\mu|^2 &=& \f{1}{1 + |\tan(\theta/2)|^2}
			\eea
			Logo, a menos de uma fase $e^{ia}$, 
			\bea
				\mu =  &=& \f{1}{\sqrt{1 + |\tan(\theta/2)|^2}} \\
				\lambda  &=& \f{\tan(\theta/2)}{\sqrt{1 + |\tan(\theta/2)|^2}}
			\eea
			Vamos supor, daqui em diante, $\theta\in[0, \pi]$ para tirarmos a tangente do módulo. Usando a solução geral de $c_+(t)$, sabemos que precisa valer
			\be
				A_+ + B_+ = 0
			\ee
			
		\end{multianswer}
		\end{exercises}
	\end{exercise}
	
	\begin{exercise}[4.4.5]
		Let $|\varphi(0)\rangle$ represent the state vector at time $t=0$ of an unstable particle, or more generally that of an unstable quantum state such as an atom in an excited state, and let $\mathrm{p}(t)$ be the probability (survival probability) that it has not decayed at time $t$. The particle is assumed to be isolated from external influences (but not from quantized fields), so that the Hamiltonian $H$ that governs the decay is time-independent. Let $|\Psi(t)\rangle$ be the state vector at time $t$ of the full quantum system
		$$
		|\Psi(t)\rangle=\exp \left(-\frac{i H t}{\hbar}\right)|\varphi(0)\rangle .
		$$
		The probability amplitude for finding the state of the quantum system at time $t$ in $|\varphi(0)\rangle$ is
		$$
		c(t)=\langle\varphi(0) \mid \Psi(t)\rangle=\left\langle\varphi(0)\left|\exp \left(-\frac{\mathrm{i} H t}{\hbar}\right)\right| \varphi(0)\right\rangle
		$$
		and the survival probability is
		$$
		\mathrm{p}(t)=|c(t)|^{2}=|\langle\Psi(t) \mid \varphi(0)\rangle|^{2}=\langle\Psi(t)|\mathcal{P}| \Psi(t)\rangle
		$$
		where $\mathcal{P}=|\varphi(0)\rangle\langle\varphi(0)|$ is the projector on the initial state.
		\begin{exercises}
			% Part 1
			\item Let us first restrict ourselves to very short times. Show that for $t \rightarrow 0$
			$$
			\mathrm{p}(t) \simeq 1-\frac{(\Delta H)^{2}}{\hbar^{2}} t^{2}
			$$
			so that, for very short times, the decay law is certainly not exponential. The expectation values of $H$ and $H^{2}$ are computed in the state $|\varphi(0)\rangle$. Note that $\Delta H$ must be finite, otherwise $|\varphi(0)\rangle$ would not belong to the domain of $H^{2}$, which would be difficult to imagine physically (see Chapter 7 for the definition of the domain of an operator).
			\begin{multianswer}
				Começamos lembrando que
				\be
					(\Delta H)^2 = \braket{H^2} - \braket{H}^2
				\ee
				Para $t\to0$, podemos aproximar
				\be
					e^{-iHt/\hbar} \approx 1 -\f{iHt}{\hbar} - \f{H^2t^2}{2\hbar^2}
				\ee
				Logo, 
				\bea
					c(t) &=& \braket{\phi_0 | 1 - \f{iHt}{\hbar} - \f{H^2t^2}{2\hbar^2} | \phi_0} \\
						&=& \braket{\phi_0|\phi_0} - \f{it}{\hbar}  \braket{H} - \f{t^2\braket{H^2}}{2\hbar^2}\\
						&=& 1 - \f{it\braket{H}}{\hbar} - \f{t^2\braket{H^2}}{2\hbar^2}
				\eea
				Portanto,
				\bea
					p(t) &=& |c(t)|^2 \\
						&=& 1  - \f{t^2\braket{H^2}}{\hbar^2}  + \f{t^2\braket{H}^2}{\hbar^2} + \f{t^4\braket{H^2}^2}{\hbar^4} \\
						&=& 1 - \f{t^2}{\hbar^2}\l(\braket{H^2}  - \braket{H}^2 \r) + \mathcal{O}(t^4)  \\
						&=& 1 - \f{t^2(\Delta H)^2}{\hbar^2}
				\eea
				como queríamos mostrar. 
			\end{multianswer}
		
			% Part 2
			\item A more general result is obtained as follows. Show first that
			$$
			\Delta \mathcal{P}^{2}=\langle\mathcal{P}\rangle-\langle\mathcal{P}\rangle^{2}
			$$
			and use (4.27) to deduce the inequality $\left(\Delta H=\left(\left\langle H^{2}\right\rangle-\langle H\rangle^{2}\right)^{1 / 2}\right)$
			$$
			\left|\frac{\mathrm{d} \mathrm{p}(t)}{\mathrm{d} t}\right| \leq \frac{2 \Delta H}{\hbar} \sqrt{\mathrm{p}(1-\mathrm{p})}
			$$
			Integrating this differential equation, derive
			$$
			\mathrm{p}(t) \geq \cos ^{2}\left(\frac{t \Delta H}{\hbar}\right) \quad 0 \leq t \leq \frac{\pi \hbar}{2 \Delta H}
			$$		]
			\begin{multianswer}
				Sabemos que $\mathcal{P} = \ket{\psi_0}\bra{\psi_0}$. Assim, como $\mathcal{P}$ é um projetor, é trivial que seja idempotente:
				\be
					\mathcal{P}^2 = \l(\ket{\psi_0}\bra{\psi_0}\r)\l(\ket{\psi_0}\bra{\psi_0}\r) = \ket{\psi_0}\braket{\psi_0|\psi_0}\bra{\psi_0} = \ket{\psi_0}\bra{\psi_0} = \mathcal{P}
				\ee
				À partir disso, decorre diretamente a propriedade de:
				\be
					(\Delta\mathcal{P})^2 = \braket{\mathcal{P}} - \braket{\mathcal{P}}^2
				\ee
				A Eq.(4.27) diz que
				\be
					\Delta_{\varphi} H \Delta_{\varphi} A \geq \frac{1}{2}\left|\langle[A, H]\rangle_{\varphi}\right|=\frac{1}{2} \hbar\left|\frac{\mathrm{d}}{\mathrm{d} t}\langle A\rangle_{\varphi}(t)\right|
				\ee
				onde, no nosso caso, teremos $A=\mathcal{P}$. Usando $p(t)=\braket{\mathcal{P}}$ e aplicando para o nosso caso:
				\be
					\l|\f{dp}{dt}\r| \leq \f{2\Delta H}{\hbar}\Delta\mathcal{P} 
				\ee
				Então, basta escrever $\Delta\mathcal{P}$ na forma desejada. Ora, isso é fácil, pois
				\bea
					\Delta\mathcal{P} &=& \sqrt{\braket{\mathcal{P}} - \braket{\mathcal{P}}^2} \\
							&=& \sqrt{p - p^2} \\
							&=& \sqrt{p(1-p)}
				\eea
				Portanto, 
				\be
					\l|\f{dp}{dt}\r| \leq \f{2\Delta H}{\hbar}\sqrt{p(1-p)} 
				\ee
				Vamos, agora, integrar essa inequação:
				\bea
					\f{dp}{\sqrt{p(1-p)}} &\leq& \f{2\Delta H}{\hbar} dt  \\
					\int \f{dp}{\sqrt{p(1-p)}} &\leq& \int \f{2\Delta H}{\hbar} dt = \f{2t\Delta H}{\hbar}
				\eea
				Vejamos o lado esquerdo com mais cuidado. Vamos usar a substituição $u=\sqrt{p}$, onde $2du = dp/\sqrt{p}$. Assim, 
				\be
					\int \f{dp}{\sqrt{p(1-p)}} = \int \f{2du}{\sqrt{1-u^2}} = 2\acos(u)
				\ee
				Agora, notamos que isso implica
				\be
					0 \leq \f{2t\Delta H}{\hbar} \leq \pi
				\ee
				Ou,
				\be
					0 \leq t \leq \frac{\pi \hbar}{2 \Delta H}
				\ee
				Voltando à inequação das integrais e usando que $u=\sqrt{p}$. 
				\bea
					\acos(u) &\leq& \f{t\Delta H}{\hbar} \\ 
					\sqrt{p} &\geq& \cos\l(\f{t\Delta H}{\hbar}\r)  \\
					p(t) &\geq& \cos^2\l(\f{t\Delta H}{\hbar}\r) 
				\eea
				o sinal da inegualdade muda, pois $\cos(x)$ decresce no intervalo $x\in[0, \pi]$. 	
			\end{multianswer}
		
			% Part 3
			\item Let $|n\rangle$ be a complete set of eigenstates of the Hamiltonian
			$$
			H|n\rangle=E_{n}|n\rangle
			$$
			Show that $c(t)$ is given by the Fourier transform of a spectral function $w(E)$
			$$
			w(E)=\sum_{n}|\langle n \mid \varphi(0)\rangle|^{2} \delta\left(E-E_{n}\right)
			$$
			Set $E_{0}=\langle H\rangle$ and give the expression of $(\Delta H)^{2}$ in terms of $w(E)$ and $E_{0}$.
			\begin{multianswer}
				Por definição,
				\be
					c(t)=\braket{\varphi(0) | \Psi(t)}
				\ee
				com
				\be
					|\Psi(t)\rangle=\exp \left(-\frac{i H t}{\hbar}\right)|\varphi(0)\rangle
				\ee
				
			\end{multianswer}
		
			% Part 4
			\item If $w(E)$ has a Lorentzian shape
			$$
			w(E)=\frac{\Gamma \hbar}{2 \pi} \frac{1}{\left(E-E_{0}\right)^{2}+\hbar^{2} \Gamma^{2} / 4}
			$$
			show that
			$$
			c(t)=\mathrm{e}^{-\mathrm{i} E_{0} t / \hbar} \mathrm{e}^{-\Gamma t / 2}
			$$
			and that the decay law is an exponential. The width of $w(E)$ is $\hbar \Gamma$, but $\Delta H$ is infinite, Thus $\Delta H$ is a rather poor measure of energy spread, and the width $\hbar \Gamma=\Delta E$ is the physically relevant quantity.
			\begin{multianswer}[true]
				
			\end{multianswer}
			
		\end{exercises}
	
		
		
	\end{exercise}
	

\end{document}
