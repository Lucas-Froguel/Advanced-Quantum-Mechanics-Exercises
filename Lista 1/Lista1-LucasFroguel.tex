\documentclass[12pt]{article}
\usepackage{amsfonts}
\usepackage{amsthm}
\usepackage{amsmath}
\usepackage[mathscr]{euscript}
\usepackage{array}
\usepackage[thinlines]{easytable}
\usepackage{tikz}
\usepackage{pgfplots}
\usepackage[margin=2cm]{geometry}
\usepackage{graphicx}
\usepackage{xcolor}
\usepackage[utf8]{inputenc}
\usepackage[T1]{fontenc}
\usepackage[portuguese]{babel}
\usepackage{braket}
\usepackage[backend=biber, style=numeric]{biblatex} %Imports biblatex package
\addbibresource{references.bib} %Import the bibliography file

\usepackage{lipsum} % Pode tirar esse :D


\def\be{\begin{equation}}
	\def\ee{\end{equation}}
\def\bee{\begin{equation*}}
	\def\eee{\end{equation*}}
\def\bea{\begin{eqnarray*}}
	\def\eea{\end{eqnarray*}}
\def\beaa{\begin{eqnarray}}
	\def\eeaa{\end{eqnarray}}

\def\bl{\begin{lemma}}
	\def\el{\end{lemma}}
\def\bt{\begin{theorem}}
	\def\et{\end{theorem}}
\def\bc{\begin{corollary}}
	\def\ec{\end{corollary}}

\def\bp{\begin{proof}}
	\def\ep{\end{proof}}

\def\bd{\begin{definition}}
	\def\ed{\end{definition}}
\def\br{\begin{remark}}
	\def\er{\end{remark}}

\def\bex{\begin{exercise}}
	\def\eex{\end{exercise}}
\def\bexa{\begin{example}}
	\def\eexa{\end{example}}
\def\bm{\begin{method}}
	\def\em{\end{method}}

\def\benu{\begin{enumerate}}
	\def\eenu{\end{enumerate}}
\def\bpr{\begin{proposition}}
	\def\epr{\end{proposition}}

\def\f{\frac}
\def\del{\partial}


\def\R{\mathbb{R}}
\def\K{\mathbb{K}}
\def\C{\mathbb{C}}
\def\I{\mathbb{I}}
\def\Z{\mathbb{Z}}
\def\Q{\mathbb{Q}}
\def\N{\mathbb{N}}

\def\cd{\cdot}

\def\v#1{{\boldsymbol{#1}}}
\def\ve#1{\hat{\boldsymbol#1}}

\def\l{\left}
\def\r{\right}
\def\la{\l\langle}
\def\ra{\r\rangle}
\def\div{\nabla\cdot}
\def\curl{\nabla\times}
\def\grad{\nabla}
\def\lap{\nabla^2}
\def\e{\varepsilon}

\def\s{\quad}
\def\ss{\qquad}
\def\infi{\infty}
\def\p{\partial}
\def\u{\cup}%union of two sets
\def\i{\cap}%intersection of two sets
\def\ds{\oplus}


\newtheorem{exercise}{Exercise}
\newtheorem{answer}{Answer}



\newcommand\norm[1]{\left\lVert#1\right\rVert}

\DeclareMathOperator{\atan}{atan}
\DeclareMathOperator{\sech}{sech}
\DeclareMathOperator{\csch}{csch}
\DeclareMathOperator{\asinh}{asinh}
\DeclareMathOperator{\atanh}{atanh}
\DeclareMathOperator{\acoth}{acoth}
\DeclareMathOperator{\acosh}{acosh}
\DeclareMathOperator{\acsch}{acsch}
\DeclareMathOperator{\asech}{asech}
\DeclareMathOperator{\im}{Im}
\DeclareMathOperator{\D}{D}
\DeclareMathOperator{\Tr}{tr}
\DeclareMathOperator{\graph}{graph}
\DeclareMathOperator{\spann}{span}
\DeclareMathOperator{\aut}{Aut}
\DeclareMathOperator{\res}{Res}
%\DeclareMathOperator{\det}{det}
%\DeclareMathOperator{\dim}{dim}

\title{Mecânica Quântica Avançada \\ Lista 1}
\author{Lucas Froguel \\ IFT}
\date{}

\begin{document}
	\maketitle
	
	
	\begin{exercise}[4.1]
		Show that a necessary and sufficient condition for $|\psi\rangle$ to be an eigenvector of a Hermitian
		operator A is that the dispersion (4.8) $\Delta_\psi A = 0$.
	\end{exercise}
	\begin{answer}
		Vamos iniciar mostrando que se $\ket{\psi}=0$, então $\Delta_\psi A = 0$. Ora, por definição:
		\be
			\Delta_\psi A = \braket{\psi|A^2|\psi} - \braket{\psi|A|\psi}^2
		\ee
		É fácil ver que
		\be
			\braket{\psi|A^2|\psi} = \bra{\psi|A^\dagger}\ket{A|\psi} = a^2
		\ee
		E também
		\be
			\braket{\psi|A|\psi}^2 = \l(a\r)^2 = a^2
		\ee
		Logo, é trivial que $\Delta_\psi A = 0$. 
		
		Agora vamos assumir que a dispersão é nula, ou seja, $\Delta A = 0$. Então ,por definição:		
		\begin{align}
			0 &= \sqrt{\langle A^2 \rangle - \langle A \rangle^2} \\
			\Rightarrow \langle A^2 \rangle &= \langle A \rangle^2
		\end{align}		
				
	\end{answer}

	\begin{exercise}[4.4.2 - 1]
		 Let $\ket{\psi}$ be a vector (not normalized) in the Hilbert space of states and H be a Hamiltonian. The
		expectation value $\braket{H}_\psi$ is
		\be
			\braket{\psi}_\psi = \f{\braket{\psi|H|\psi}}{\braket{\psi|\psi}}
		\ee
		Show that if the minimum of this expectation value is obtained for $\ket{\psi}=\ket{\psi_m}$ and the maximum for $\ket{\psi}=\ket{\psi_M}$ , then
		\be
			H\ket{\psi_m} = E_m\ket{\psi_m}, \quad\quad H\ket{\psi_M} = E_M\ket{\psi_M}
		\ee 	
		where $E_m$ and $E_M$ are the smallest and largest eigenvalues.
	\end{exercise}
	\begin{answer}
		É evidente que
		\be
			\braket{H}_{\psi_m} = \f{\braket{\psi_m|H|\psi_m}}{\braket{\psi_m|\psi_m}} = E_m
		\ee
		Portanto, é evidente que se $\braket{H}_{\psi_m}$ for mínimo, então $E_m$ também é. Vale um raciocínio análogo para $E_M=\braket{H}_{\psi_M}$. 		
	\end{answer}
	
	\begin{exercise}[4.4.2 - 2]
		We assume that the vector $|\varphi\rangle$ depends on a parameter $\alpha:|\varphi\rangle=|\varphi(\alpha)\rangle$. Show that if
		\begin{equation}
			\frac{\partial\braket{H}_{\varphi(\alpha)}}{\partial \alpha}\bigg|_{\alpha=\alpha_0}=0,
		\end{equation}
		then $E_{m} \leq\langle H\rangle_{\varphi\left(\alpha_{0}\right)}$ if $\alpha_{0}$ corresponds to a minimum of $\langle H\rangle_{\varphi(\alpha)}$, and $\langle H\rangle_{\varphi\left(\alpha_{0}\right)} \leq E_{M}$ if $\alpha_{0}$ corresponds to a maximum. This result forms the basis of an approximation method called the variational method (Section 14.1.4).
	\end{exercise}
	\begin{answer}
		Vamos abrir a derivada:
		\be
			\frac{\partial\braket{H}{\varphi(\alpha)}}{\partial \alpha} = \f{1}{\braket{\psi|\psi}}\l( \bra{\psi|H} \partial_\alpha\ket{\psi} + (\partial_\alpha\bra{\psi})\ket{H|\psi}\r) - \f{\braket{\psi|H|\psi}}{\braket{\psi|\psi}^2}\l( (\partial_\alpha\bra{\psi})\ket{\psi} + \bra{\psi}\partial_\alpha\ket{\psi}\r) 
		\ee
		Então, em $\alpha_0$
		\be
			\braket{\psi|H|\partial_\alpha\psi} + \braket{\partial_\alpha\psi|H|\psi} = \f{\braket{\psi|H|\psi}}{\braket{\psi|\psi}}\big( \braket{\partial_\alpha\psi|\psi} + \braket{\psi|\partial_\alpha\psi}\big)
		\ee
		Isolando a quantidade de interesse:
		\be
			\braket{H}_{\varphi(\alpha_0)} = \f{\braket{\psi|H|\partial_\alpha\psi} + \braket{\partial_\alpha\psi|H|\psi}}{\braket{\partial_\alpha\psi|\psi} + \braket{\psi|\partial_\alpha\psi}}
		\ee 
		Podemos reescrever, usando que $H=H^\dagger$:
		\be
			\braket{H}_{\varphi(\alpha_0)} = \f{\braket{\partial_\alpha\psi|H|\psi}^\dagger + \braket{\partial_\alpha\psi|H|\psi}}{\braket{\partial_\alpha\psi|\psi}^\dagger + \braket{\partial_\alpha\psi|\psi}}
		\ee
		
		Podemos expandir qualquer estado usando os autokets do hamiltoniano, que são ortonormais:
		\be
			\ket{\psi} = \sum c_j\ket{\psi_j}
		\ee
		De modo que
		\be
			\partial_\alpha\ket{\psi} = \sum \f{\partial c_j}{\partial\alpha} \ket{\psi_j}
		\ee
		Assim, vale que
		\be
			\braket{\partial_\alpha\psi|\psi} = \sum c_j\f{\partial c_j^*}{\partial\alpha}
		\ee
		E também
		\be
			\braket{\partial_\alpha\psi|H|\psi} = \sum E_jc_j\f{\partial c_j^*}{\partial\alpha}
		\ee
		Considere o denominador:
		\bea
			\braket{\partial_\alpha\psi|\psi}^\dagger + \braket{\partial_\alpha\psi|\psi} 
				&=& \sum c_j \f{\partial c_j^*}{\partial\alpha} + c_j^*\f{\partial c_j}{\partial\alpha} \\
				&=& \sum \f{\partial}{\partial\alpha} (c_j^*c_j) \\
				&=& \sum \partial_\alpha |c_j|^2 \\
				&=& \partial_\alpha \sum |c_j|^2
		\eea
		Considerando, enfim, o numerador e fazendo os mesmos cálculos:
		\bea
			\braket{\partial_\alpha\psi|H|\psi}^\dagger + \braket{\partial_\alpha\psi|H|\psi} 
				&=& \sum E_jc_j\f{\partial c_j^*}{\partial\alpha} + E_j c_* \f{\partial c_j}{\partial\alpha} \\
				&=& \sum E_j \partial_\alpha |c_j|^2 \\
				&=& \partial_\alpha \sum E_j|c_j|^2 
		\eea
		É óbvio, então, que
		\be
			E_m \partial_\alpha\sum |c_j|^2 \leq \partial_\alpha\sum E_j|c_j|^2 \leq E_M \partial_\alpha\sum |c_j|^2
		\ee 
		Portanto, concluímos que
		\be
			E_m \leq \braket{H}_{\varphi(\alpha_0)} \leq E_M
		\ee
		
		
	\end{answer}


\end{document}
