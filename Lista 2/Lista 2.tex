\documentclass[12pt]{article}
\usepackage{amsfonts}
\usepackage{amsthm}
\usepackage{amsmath}
\usepackage[mathscr]{euscript}
\usepackage{array}
\usepackage[thinlines]{easytable}
\usepackage{tikz}
\usepackage{pgfplots}
\usepackage[margin=2cm]{geometry}
\usepackage{graphicx}
\usepackage{xcolor}
\usepackage[utf8]{inputenc}
\usepackage[T1]{fontenc}
\usepackage[portuguese]{babel}
\usepackage{braket}
\usepackage{thmtools} 
\usepackage{hyperref}
\usepackage{environ}
\usepackage{enumitem}
\usepackage[backend=biber, style=numeric]{biblatex} %Imports biblatex package
\addbibresource{references.bib} %Import the bibliography file
%
\usepackage{lipsum} % Pode tirar esse :D

\graphicspath{.Images/}

\hypersetup{
	colorlinks=true,
	linkcolor=blue,
	filecolor=magenta,      
	urlcolor=cyan,
	pdftitle={Quântica Avançada - L1 - Lucas Froguel}
}


\def\be{\begin{equation}}
	\def\ee{\end{equation}}
\def\bee{\begin{equation*}}
	\def\eee{\end{equation*}}
\def\bea{\begin{eqnarray*}}
	\def\eea{\end{eqnarray*}}
\def\beaa{\begin{eqnarray}}
	\def\eeaa{\end{eqnarray}}

\def\f{\frac}
\def\del{\partial}

\def\R{\mathbb{R}}
\def\K{\mathbb{K}}
\def\C{\mathbb{C}}
\def\I{\mathbb{I}}
\def\Z{\mathbb{Z}}
\def\Q{\mathbb{Q}}
\def\N{\mathbb{N}}

\def\cd{\cdot}

\def\v#1{{\boldsymbol{#1}}}
\def\ve#1{\hat{\boldsymbol#1}}

\def\l{\left}
\def\r{\right}
\def\la{\l\langle}
\def\ra{\r\rangle}
\def\div{\nabla\cdot}
\def\curl{\nabla\times}
\def\grad{\nabla}
\def\lap{\nabla^2}

\def\s{\quad}
\def\ss{\qquad}
\def\infi{\infty}
\def\p{\partial}
\def\u{\cup}%union of two sets
\def\i{\cap}%intersection of two sets
\def\ds{\oplus}



\newtheorem{exercise}{Exercise}
\newtheorem{partinner}{Part}[exercise]

\newlist{exercises}{enumerate}{1}
\setlist[exercises]{wide = 0pt, listparindent=\parindent,labelsep = 0pt,leftmargin =\labelwidth}
\setlist[exercises, 1]{label =\itshape \bfseries Part~\arabic*.\; }

\newenvironment{answer}{\noindent\textbf{\textit{Answer.}} \normalfont }{\par\noindent\rule{\textwidth}{0.4pt}}
\NewEnviron{multianswer}[1][false]{%
	\ifthenelse{\equal{#1}{true}}%
	{\def\rulewidth{\textwidth}}%
	{\def\rulewidth{0.7\textwidth}}%
	\\ \noindent\textbf{\textit{Answer.}} \normalfont%
	\BODY%
	\par\noindent\rule{\rulewidth}{0.1pt}%
}


\newcommand\norm[1]{\left\lVert#1\right\rVert}

\DeclareMathOperator{\atan}{atan}
\DeclareMathOperator{\cotan}{cotan}
\DeclareMathOperator{\acos}{acos}
\DeclareMathOperator{\sech}{sech}
\DeclareMathOperator{\csch}{csch}
\DeclareMathOperator{\asinh}{asinh}
\DeclareMathOperator{\atanh}{atanh}
\DeclareMathOperator{\acoth}{acoth}
\DeclareMathOperator{\acosh}{acosh}
\DeclareMathOperator{\acsch}{acsch}
\DeclareMathOperator{\asech}{asech}
\DeclareMathOperator{\D}{D}
\DeclareMathOperator{\Tr}{tr}
\DeclareMathOperator{\Res}{Res}

\title{Mecânica Quântica Avançada \\ Lista 2}
\author{Lucas Froguel \\ IFT}
\date{}

\begin{document}
	\maketitle
	\listoftheorems[title={List of Exercises}]
	
	\begin{exercise}[6.5.1 - Independence of the tensor product from the choice of basis]
		Verify that the definition (6.3) of the tensor product of two vectors is independent of the choice of basis in $\mathcal{H}_1$ and $\mathcal{H}_2$.
	\end{exercise}
	\begin{answer}
		Let $\ket{n'}$ and $\ket{m'}$ be two other basis of the Hilbert spaces one and two, respectively. Then, it is true that
		\be
			\begin{aligned}
				\ket{n} &= \sum a_{n'}\ket{n'} \\
				\ket{m} &= \sum b_{m'}\ket{m'}
			\end{aligned}	
		\ee	
		Thus, we can write
		\bea
			\ket{\varphi}\otimes\ket{\chi} &=& \sum_{n, m} c_nd_m\ket{n}\otimes\ket{m} \\ 
				&=& \sum_{n, m} c_nd_m \l(\sum_{n'} a_{n'}\ket{n'}\r) \otimes \l(\sum_{m'}b_{m'}\ket{m'} \r) \\
				&=& \sum_{n', m'} a_{n'}b_{m'} \l(\sum_n c_n \r) \l(\sum_m d_m\r) \ket{n'}\otimes\ket{m'} \\
				&=& \sum_{n', m'} e_{n'} f_{m'} \ket{n'}\otimes\ket{m'}
		\eea
		This shows that the tensor product does not depend on the choice of basis.		
	\end{answer}

	\begin{exercise}[2 - Representação matricial de produtos tensoriais]
		Calculate the tensor products of two-level systems.
	\end{exercise}
	\begin{answer}
		We can calculate the tensor products of $\ket{+}$ and $\ket{-}$ as follows:
		
		\begin{align*}
			\ket{+} \otimes \ket{-} &= \begin{pmatrix} 1 \\ 0 \end{pmatrix} \otimes \begin{pmatrix} 0 \\ 1 \end{pmatrix} \\
			&= \begin{pmatrix} 1\begin{pmatrix} 0 \\ 1 \end{pmatrix} \\ 0\begin{pmatrix} 0 \\ 1 \end{pmatrix} \end{pmatrix} \\
			&= \begin{pmatrix} 0 \\ 1 \\ 0 \\ 0 \end{pmatrix} 
		\end{align*}
		
		\begin{align*}
			\ket{-} \otimes \ket{+} &= \begin{pmatrix} 0 \\ 1 \end{pmatrix} \otimes \begin{pmatrix} 1 \\ 0 \end{pmatrix} \\
			&= \begin{pmatrix} 0\begin{pmatrix} 1 \\ 0 \end{pmatrix} \\ 1\begin{pmatrix} 1 \\ 0 \end{pmatrix} \end{pmatrix} \\
			&= \begin{pmatrix} 0 \\ 0 \\ 1 \\ 0 \end{pmatrix}
		\end{align*}
		
		\begin{align*}
			\ket{-} \otimes \ket{-} &= \begin{pmatrix} 0 \\ 1 \end{pmatrix} \otimes \begin{pmatrix} 0 \\ 1 \end{pmatrix} \\
			&= \begin{pmatrix} 0\begin{pmatrix} 0 \\ 1 \end{pmatrix} \\ 1\begin{pmatrix} 0 \\ 1 \end{pmatrix} \end{pmatrix} \\
			&= \begin{pmatrix} 0 \\ 0 \\ 0 \\ 1 \end{pmatrix} 
		\end{align*}
		
		Now for the three dimension qubits, we will write the answers directly:
		\bea
			\ket{+++} &=& \begin{bmatrix} 1 \\ 0 \\ 0 \\ 0 \\ 0 \\ 0 \\ 0 \\ 0 \\ \end{bmatrix}, \qquad
			\ket{++-} = \begin{bmatrix} 0 \\ 1 \\ 0 \\ 0 \\ 0 \\ 0 \\ 0 \\ 0 \\ \end{bmatrix} \\
			\ket{+-+} &=& \begin{bmatrix} 0 \\ 0 \\ 1 \\ 0 \\ 0 \\ 0 \\ 0 \\ 0 \\ \end{bmatrix}, \qquad
			\ket{+--} = \begin{bmatrix} 0 \\ 0 \\ 0 \\ 1 \\ 0 \\ 0 \\ 0 \\ 0 \\ \end{bmatrix} \\
			\ket{-++} &=& \begin{bmatrix} 0 \\ 0 \\ 0 \\ 0 \\ 1 \\ 0 \\ 0 \\ 0 \\ \end{bmatrix}, \qquad
			\ket{-+-} = \begin{bmatrix} 0 \\ 0 \\ 0 \\ 0 \\ 0 \\ 1 \\ 0 \\ 0 \\ \end{bmatrix} \\
			\ket{--+} &=& 	\begin{bmatrix} 0 \\ 0 \\ 0 \\ 0 \\ 0 \\ 0 \\ 1 \\ 0 \\ \end{bmatrix}, \qquad
			\ket{---} = \begin{bmatrix} 0 \\ 0 \\ 0 \\ 0 \\ 0 \\ 0 \\ 0 \\ 1 \\ \end{bmatrix}
		\eea
	\end{answer}
	
	\begin{exercise}[6.5.2]
		Write down explicitly the 4$\times$4 matrix $A\otimes B$, the tensor product of the 2$\times$2 matrices $A$ and $B$:
		\be
			A = 
			\begin{pmatrix}
				a & b \\
				c & d
			\end{pmatrix}, \qquad
			B = 
			\begin{pmatrix}
				\alpha & \beta \\ 
				\gamma & \delta 
			\end{pmatrix}
		\ee
	\end{exercise}
	\begin{answer}
		It is very easy do perform this calculation:
		\be
			A \otimes B = 
			\begin{pmatrix}
				a\alpha & a\beta & b\alpha & b\beta \\
				a\gamma & a\delta & b\gamma & b\delta \\
				c\alpha & c\beta & d\alpha & d\beta \\
				c\gamma & c\delta & d\gamma & d\delta \\
			\end{pmatrix}.
		\ee
		We just multiply each element of the first matrix by the whole second matrix. 
	\end{answer}
	
	\begin{exercise}[6.5.3 - Properties of state operators]
		\begin{exercises}
			\item Show that $\rho_{ii}\geq 0$, $\rho_{jj}\geq 0$, and $\det{A} \geq 0$, from which $|\rho_{ij}|^2 \leq \rho_{ii} \rho_{jj}$ . Also deduce that if $\rho_{ii} = 0$, then $\rho_{ij} = \rho_{ji}^* = 0$.
			\begin{multianswer}
				We can always write
				\be
					\rho = \sum a_n \ket{\phi_n}\bra{\phi_n}
				\ee
				for some states $\ket{\phi_n}$ and $a_n\geq 0$. Thus, the diagonal matrix elements are
				\bea
					\rho_{ii} &=& \bra{\phi_i} \l(\sum a_n \ket{\phi_n}\bra{\phi_n}\r) \ket{\phi_i} \\
						&=& a_i 
				\eea
				Hence, $\rho_{ii}\geq0$. We also note that
				\be
					\det{A} = \rho_{ii}\rho_{jj} - |\rho_{ij}|^2
				\ee
				where we used the fact that $A$ is hermitian. This implies that
				\be
					\det{A} \geq 0 \iff |\rho_{ij}|^2 \leq \rho_{ii}\rho_{jj}
				\ee
				Using this inequality, if $\rho_ii=0$, then
				\be
					0 \leq \rho_{ij}\rho_{ji} \leq 0
				\ee
				Obviously, $\rho_{ij}=0$ or $\rho_{ji}=0$, but is does not matter, because they are the complex conjugate of each other, so if one is zero, the other is zero as well.				
			\end{multianswer}
			
			% Part 2
			\item Show that if there exists a maximal test giving 100$\%$ probability for the quantum state described
			by a state operator $\rho$, then this state is a pure state. Also show that if $\rho$ describes a pure state,
			and if it can be written as
			\be
				\rho = \lambda\rho' + (1 - \lambda)\rho'', 0 \leq \lambda 
				\leq 1
			\ee
			then $\rho=\rho'=\rho''$ . Hint: first demonstrate that if $\rho'$ and $\rho''$ are generic state operators, then $\rho$ is
			a state operator. The state operators form a convex subset of Hermitian operators.
			\begin{multianswer}[true]
				
			\end{multianswer}
			
		\end{exercises}
	\end{exercise}
	









\end{document}