\documentclass[12pt]{article}
\usepackage{amsfonts}
\usepackage{amsthm}
\usepackage{amsmath}
\usepackage[mathscr]{euscript}
\usepackage{array}
\usepackage[thinlines]{easytable}
\usepackage{tikz}
\usepackage{pgfplots}
\usepackage[margin=2cm]{geometry}
\usepackage{graphicx}
\usepackage{xcolor}
\usepackage[utf8]{inputenc}
\usepackage[T1]{fontenc}
\usepackage[portuguese]{babel}
\usepackage{braket}
\usepackage{thmtools} 
\usepackage{hyperref}
\usepackage{environ}
\usepackage{enumitem}
\usepackage[backend=biber, style=numeric]{biblatex} %Imports biblatex package
\addbibresource{references.bib} %Import the bibliography file
\usetikzlibrary{angles,quotes}

\usepackage{lipsum} % Pode tirar esse :D

\graphicspath{.Images/}

\hypersetup{
	colorlinks=true,
	linkcolor=blue,
	filecolor=magenta,      
	urlcolor=cyan,
	pdftitle={Quântica Avançada - P1 - Lucas Froguel}
}


\def\be{\begin{equation}}
	\def\ee{\end{equation}}
\def\bee{\begin{equation*}}
	\def\eee{\end{equation*}}
\def\bea{\begin{eqnarray*}}
	\def\eea{\end{eqnarray*}}
\def\beaa{\begin{eqnarray}}
	\def\eeaa{\end{eqnarray}}

\def\f{\frac}
\def\del{\partial}

\def\R{\mathbb{R}}
\def\K{\mathbb{K}}
\def\C{\mathbb{C}}
\def\I{\mathbb{I}}
\def\Z{\mathbb{Z}}
\def\Q{\mathbb{Q}}
\def\N{\mathbb{N}}

\def\cd{\cdot}

\def\v#1{{\boldsymbol{#1}}}
\def\ve#1{\hat{\boldsymbol#1}}

\def\l{\left}
\def\r{\right}
\def\la{\l\langle}
\def\ra{\r\rangle}
\def\div{\nabla\cdot}
\def\curl{\nabla\times}
\def\grad{\nabla}
\def\lap{\nabla^2}

\def\s{\quad}
\def\ss{\qquad}
\def\infi{\infty}
\def\p{\partial}
\def\u{\cup}%union of two sets
\def\i{\cap}%intersection of two sets
\def\ds{\oplus}



\newtheorem{exercise}{Exercício}
\newtheorem{partinner}{Item}[exercise]

\newlist{exercises}{enumerate}{1}
\setlist[exercises]{wide = 0pt, listparindent=\parindent,labelsep = 0pt,leftmargin =\labelwidth}
\setlist[exercises, 1]{label =\itshape \bfseries Item~\arabic*.\; }

\newenvironment{answer}{\noindent\textbf{\textit{Resposta.}} \normalfont }{\par\noindent\rule{\textwidth}{0.4pt}}
\NewEnviron{multianswer}[1][false]{%
	\ifthenelse{\equal{#1}{true}}%
	{\def\rulewidth{\textwidth}}%
	{\def\rulewidth{0.7\textwidth}}%
	\\ \noindent\textbf{\textit{Resposta.}} \normalfont%
	\BODY%
	\par\noindent\rule{\rulewidth}{0.1pt}%
}


\newcommand\norm[1]{\left\lVert#1\right\rVert}

\DeclareMathOperator{\atan}{atan}
\DeclareMathOperator{\cotan}{cotan}
\DeclareMathOperator{\acos}{acos}
\DeclareMathOperator{\asin}{asin}
\DeclareMathOperator{\sech}{sech}
\DeclareMathOperator{\csch}{csch}
\DeclareMathOperator{\asinh}{asinh}
\DeclareMathOperator{\atanh}{atanh}
\DeclareMathOperator{\acoth}{acoth}
\DeclareMathOperator{\acosh}{acosh}
\DeclareMathOperator{\acsch}{acsch}
\DeclareMathOperator{\asech}{asech}
\DeclareMathOperator{\D}{D}
\DeclareMathOperator{\tr}{tr}
\DeclareMathOperator{\Res}{Res}

\title{Mecânica Quântica Avançada \\ Prova 1}
\author{Lucas Froguel \\ IFT}
\date{}

\begin{document}
	\maketitle
	\listoftheorems[title={Prova 1}]

	\begin{exercise}
		Considere uma molécula constituída de três átomos idênticos nos
		sítios (vértices) de um triângulo equilátero, conforme mostrado na figura. Vamos considerar que o íon desta molécula é formado adicionando-se um único elétron a ela; como vimos em aula, deve haver um efeito de delocalização, e o elétron deve se acomodar em um estado distribuído ao redor dos sítios. Suponha que o elemento de matriz da Hamiltoniana para o elétron em dois sítios adjacentes $i, j$ é $\braket{i|H|j} = -a, a>0$, para $i\not=j$; por simetria, os elementos diagonais são todos	iguais, $\braket{i|H|i}=E_0$.
		\begin{exercises}
			\item Escreva a Hamiltoniana na forma matricial e calcule os níveis de energia.
			\begin{multianswer}
				A hamiltoniana é
				\be
					H = 
					\begin{pmatrix}
						E_0 & -a & -a \\
						-a & E_0 & -a \\
						-a & -a & E_0 \\
					\end{pmatrix}
				\ee
				Diagonalizando a matriz (processo feito em aula), achamos que as autoenergias são
				\be
					E_s = E_0 -2a\cos\l( \f{2\pi s}{3}\r)
				\ee
				Assim, nossas energias são
				\be
					E_0 = E_0 - 2a, \quad E_1 = E_0 -2a\cos\l(\f{2\pi}{3}\r), \quad E_2 = E_0 -2a\cos\l(\f{4\pi}{3}\r)
				\ee
			\end{multianswer}
			
			% Part 2
			\item Suponha que um campo elétrico na direção $z$ é aplicado, de modo que a energia potencial para o elétron na posição rotulada por “1” diminui por uma parcela $b$
			$(b>0)$. Calcule os níveis de energia e os autoestados de energia.
			\begin{multianswer}
				Agora quebramos a simetria que fazia com que os elementos das diagonais fossem $E_0$, pois podemos distinguir o estado 1 dos estados 2 e 3. Logo, a hamiltoniana modificada é 
				\be
				H = 
				\begin{pmatrix}
					E_0 - b & -a & -a \\
					-a & E_0 + b & -a \\
					-a & -a & E_0 + b \\
				\end{pmatrix}
				\ee
				Achar os autovalores e vetores de uma matriz $3\times3$ é um processo matemático bem definido e metódico. Fazendo essa operação, obtemos
				\bea
					E_1 = E_0 + a + b, &\quad& \ket{1'} = \f{-\ket{2} + \ket{3}}{\sqrt{2}} \\
					E_2 = E_0 - \f{1}{2}\l( a + \sqrt{9a^2 - 4ab + 4b^2}\r), &\quad& \ket{2'} = \f{c_2\ket{1} +\ket{2} + \ket{3}}{\sqrt{2+c_2^2}} \\
					E_3 = E_0 + \f{1}{2}\l( -a + \sqrt{9a^2 - 4ab + 4b^2}\r), &\quad& \ket{3'} = \f{c_3\ket{1} +\ket{2} + \ket{3}}{\sqrt{2+c_3^2}}
				\eea
				onde
				\bea
					c_2 &=& - \f{a -2b - \sqrt{9a^2 -4ab + 4b^2}}{2a} \\					c_2 &=& - \f{a -2b + \sqrt{9a^2 -4ab + 4b^2}}{2a} 
				\eea
			\end{multianswer}
		
			% Part 3
			\item Suponha que, no caso do item (b), o elétron está no estado fundamental.
			Quase instantaneamente, o campo roda de $120^\circ$ e passa a apontar na direção
			rotulada como “2”. Calcule a probabilidade do elétron permanecer no estado fundamental.
			\begin{multianswer}[true]
				Nesse caso, o estado 2 ocupa o antigo estado 1, o estado 3 ocupa o antigo 2 e o estado 1 o antigo 3. Notamos que o estado fundamental original é $\ket{2'}$ e o novo estado fundamental é $\ket{3'}$. Portanto
				\bea
					p &=& \l| \braket{3' | 2'} \r|^2 \\
						&=& \l(\f{2 + 2c_2c_3}{\sqrt{(2+c_2^2)(2+c_3^2)}}\r)^2 \\
						&=& \f{4 + 8c_2c_3 + 4c_2^2c_3^2}{4 + 2(c_2^2+c_3^2)+c_2^2c_3^2}
				\eea
			\end{multianswer}
			
		\end{exercises}
	\end{exercise}
	
	\begin{exercise}
		Para cada estado $\ket{\psi}$ de dois spins $1/2$ abaixo, escrito em termos dos autovetores de $S_z$, diga se o estado é emaranhado ou não (justificando sua 		resposta), calcule seu operador estado (matriz densidade) explicitamente, e calcule
		o operador estado reduzido referente à particula 1 (na notação $\ket{1, 2}$).
		\begin{enumerate}
			\item $ \ket{\psi} = \f{1}{\sqrt{2}} (\ket{+-} - \ket{-+}) $
			\begin{multianswer}
				Logo de cara vemos que esse é um estado de Bell, um estado que é notoriamente emaranhado. Isso é verdade, pois é impossível escrever esse estado como um produto tensorial disjunto de dois estados de $\ket{1}$ e de $\ket{2}$. O operador de densidade é
				\bea
					\rho &=& \ket{\psi}\bra{\psi} \\
						&=& \f{1}{2} (\ket{+-} - \ket{-+}) (\bra{+-} - \bra{-+}) \\
						&=& \f{1}{2}\Big( \ket{+-}\bra{+-} - \ket{+-}\bra{-+} - \ket{-+}\bra{+-} + \ket{-+}\bra{-+} \Big) \\
						&\dot{=}& \f{1}{2}
						\begin{pmatrix}
							0 & 0 & 0 & 0 \\
							0 & 1 & -1 & 0 \\
							0 & -1 & 1 & 0 \\
							0 & 0 & 0 & 0 \\
						\end{pmatrix}
				\eea
				onde usamos a sequência $\{ \ket{++}, \ket{+-}, \ket{-+}, \ket{--} \}$. Tomando o traço parcial 
				\be
					\rho_1 = \tr_2(\rho) = \f{1}{2}
					\begin{pmatrix}
						1 & 0 \\
						0 & 1 \\
					\end{pmatrix}
				\ee
			\end{multianswer}
			
			\item $ \ket{\psi} = \ket{++} $
			\begin{multianswer}
				O estado é claramente não emaranhado, pois podemos escrevê-lo como 
				\be
					\ket{\psi} = \ket{+} \otimes \ket{+}
				\ee
				Seu operador de estado é
				\be
					\rho = \ket{++}\bra{++} = 
					\begin{pmatrix}
						1 & 0 & 0 & 0 \\
						0 & 0 & 0 & 0 \\
						0 & 0 & 0 & 0 \\
						0 & 0 & 0 & 0 \\
					\end{pmatrix}
				\ee
				e o operador de estado reduzido é
				\be
					\rho_1 = \tr_2\rho = 
					\begin{pmatrix}
						1 & 0 \\
						0 & 0 \\
					\end{pmatrix}
				\ee
			\end{multianswer}
			
			\item $ \ket{\psi} = \f{1}{5\sqrt{2}} \big( \ket{++} + 3i\ket{+-} - 2i\ket{-+} + 6\ket{--} \big) $
			\begin{multianswer}[true]
				Podemos reescrever esse estado como
				\be
					\ket{\psi} = \f{1}{5\sqrt{2}} \big( \ket{+} - 2i\ket{-}\big) \otimes \big( \ket{+} + 3i \ket{-}\big) 
				\ee
				de modo que vemos que ele não é emaranhado. Seu operador de estado é
				\bea
					\rho &=& \f{1}{50}   \\
						&=&  \frac{1}{50} 
					\begin{pmatrix}
						1 & -3i & 2i & 6 \\
						3i & 9 & -6i & -18i \\
						-2i & 6i & 4 & -12i \\
						6 & -18i & 12i & 36 \\
					\end{pmatrix}					
				\eea
				Tomando o traço parcial
				\be
					\rho_1 = \f{1}{50} 
					\begin{pmatrix}
						10 & -16i \\
						-20i & 40 \\
					\end{pmatrix} 
					= \f{1}{25} 
					\begin{pmatrix}
						5 & -8i \\
						-10i & 20 \\
					\end{pmatrix} 
				\ee
			\end{multianswer}
		\end{enumerate}
	\end{exercise}
	
	
	\begin{exercise}
		Considere um ensemble estatístico de sistemas de um spin-$1/2$. 
		
		\begin{exercises}
			\item Considere que o ensemble é um estado puro, apontando em uma direção $\hat{n}$ (ver figura), e descrito por um ket $\alpha$. Suponha que os valores esperados $\braket{S_x}$ e $\braket{S_z}$ são conhecidos, mas apenas o sinal de $\braket{S_y}$ é conhecido. Construa explicitamente a matriz densidade $2\times 2$ que descreve o sistema.
			\begin{multianswer}
				Basicamente precisamos aplicar essas condições dos valores médios e montar um sistema de equações que nos darão as restrições necessárias sobre o estado puro
				\be
					\ket{\psi} = \cos\l(\f{\beta}{2}\r)\ket{+} + e^{i\alpha}\sin\l(\f{\beta}{2}\r)\ket{-} 
				\ee
				Para $S_z$:
				\be
					\f{\hbar}{2}\l( \cos^2\l(\f{\beta}{2}\r) - \sin^2\l(\f{\beta}{2}\r)\r) = \f{\hbar}{2}\cos\beta = \braket{S_z}
				\ee
				Para $S_x$:
				\bea
					\braket{S_x} &=& \braket{\psi | S_x | \psi} \\
							&=& \f{\hbar}{2} \braket{\psi \l| \l( \ket{+}\bra{-} + \ket{-}\bra{+} \r) \r| \psi} \\
							&=& \f{\hbar}{2} \l( \cos\l(\f{\beta}{2}\r) e^{-i\alpha} \sin\l(\f{\beta}{2}\r) + \cos\l(\f{\beta}{2}\r)e^{i\alpha}\sin\l(\f{\beta}{2}\r) \r) \\
							&=& \f{1}{2}\f{\hbar}{2} \sin(\beta) \l( e^{i\alpha} + e^{-i\alpha} \r)  \\
							&=& \f{\hbar}{2} \sin\beta\cos\alpha
				\eea
				Para $S_y$:
				\bea
					\epsilon\braket{S_y} &=& \braket{\psi | S_y | \psi} \\
					&=& \f{\hbar}{2} \braket{\psi \l| \l( -\ket{+}\bra{-} + \ket{-}\bra{+} \r) \r| \psi} \\
					&=& \f{\hbar}{2} \l( -\cos\l(\f{\beta}{2}\r) e^{-i\alpha} \sin\l(\f{\beta}{2}\r) + \cos\l(\f{\beta}{2}\r)e^{i\alpha}\sin\l(\f{\beta}{2}\r) \r) \\
					&=& \f{1}{2}\f{\hbar}{2} \sin(\beta) \l( e^{i\alpha} - e^{-i\alpha} \r)  \\
					&=& \f{i\hbar}{2} \sin\beta\sin\alpha
				\eea
				onde $\epsilon=\pm 1$. Pela primeira,
				\be
					\beta = \acos\l( \f{2\braket{S_z}}{\hbar} \r)
				\ee
				Somando os quadrados da segunda e terceira expressões
				\bea
					\braket{S_x}^2 + \braket{S_y}^2 &=& \f{\hbar^2}{4} \sin^2\beta \l( \cos^2\alpha - \sin^2\alpha \r)   \\
						&=& \f{\hbar^2}{4} \sin^2\beta \cos(2\alpha) \\
					\alpha &=& \f{1}{2} \acos \l( \f{4(\braket{S_x}^2 + \braket{S_y}^2)}{\hbar^2\sin^2\beta} \r) \\
						&=& \f{1}{2} \acos \l( \f{4(\braket{S_x}^2 + \braket{S_y}^2)}{\hbar^2\l( 1 - \f{4\braket{S_z}^2}{\hbar^2}\r)} \r) \\ 
						&=& \f{1}{2} \acos \l( \f{\braket{S_x}^2 + \braket{S_y}^2}{\f{\hbar^2}{4} - \braket{S_z}^2} \r) \\
				\eea
				Assim, fomos capazes de calcular $\alpha$ e  $\beta$ explicitamente em função apenas dos valores médios dados. A matriz de densidade geral é
				\be
					\rho = 
					\begin{pmatrix}
						\cos^2(\beta/2) & e^{i\alpha} \sin(\beta) / 2 \\
						e^{-i\alpha}\sin(\beta) / 2 & \sin^2(\beta/2) \\
					\end{pmatrix}
				\ee
				que possui determinante notoriamente igual a 0, típico de estados puros, como esperado. Notamos que o que temos até agora é
				\bea
					\cos^2\l(\f{\beta}{2}\r) &=& \f{1}{2} + \f{\braket{S_z}}{\hbar} \\
					\sin^2\l(\f{\beta}{2}\r) &=& \f{1}{2} - \f{\braket{S_z}}{\hbar} \\ 
					\sin\beta &=& \sqrt{ 1 - \f{4\braket{S_z}^2}{\hbar^2}} \\
					e^{\pm i\alpha} &=& \sqrt{\f{1}{2} + \f{1}{2} \f{\braket{S_x}^2 + \braket{S_y}^2}{\f{\hbar^2}{4} - \braket{S_z}^2}} \pm i \sqrt{\f{1}{2} - \f{1}{2} \f{\braket{S_x}^2 + \braket{S_y}^2}{\f{\hbar^2}{4} - \braket{S_z}^2}}
				\eea
				Podemos melhorar isso. Subtraindo os quadrados da segunda e terceira expressões:
				\bea
					\braket{S_x}^2 - \braket{S_y}^2 &=& \f{\hbar^2}{4} \sin^2\beta \l( \cos^2\alpha + \sin^2\alpha \r)   \\
						&=& \f{\hbar^2}{4}\sin^2\beta \\
						\sin\beta &=& \f{2}{\hbar} \sqrt{\braket{S_x}^2 - \braket{S_y}^2} 
				\eea
				Ora, usando isso:
				\be
					\alpha = \f{1}{2}\acos\l( \f{\braket{S_x}^2 + \braket{S_y}^2}{\braket{S_x}^2 - \braket{S_y}^2} \r)
				\ee
				Logo,
				\be
					e^{\pm i\alpha} = \f{1}{\sqrt{2}} \sqrt{ 1 + \f{\braket{S_x}^2 + \braket{S_y}^2}{\braket{S_x}^2 - \braket{S_y}^2} } \pm i \f{1}{\sqrt{2}} \sqrt{ 1 - \f{\braket{S_x}^2 + \braket{S_y}^2}{\braket{S_x}^2 - \braket{S_y}^2} }
				\ee
				Portanto, agora temos uma expressão elegante para o produto:
				\be
					e^{\pm i\alpha}\sin\beta = \f{2}{\hbar} \l( \braket{S_x} \mp \lvert\braket{S_y}\rvert \r)
				\ee 
				Agora podemos escrever de maneira explícita a matrix:
				\be
					\rho = \f{1}{2} 
					\begin{pmatrix}
						1 & 0 \\
						0 & 1 \\ 
					\end{pmatrix}
					+ \f{1}{\hbar} 
					\begin{pmatrix}
						\braket{S_z} & \braket{S_x} - \lvert\braket{S_y}\rvert \\
						\braket{S_x} +\lvert\braket{S_y}\rvert & -\braket{S_z} \\
					\end{pmatrix}
				\ee
				com a restrição de que $\det\rho=0$. Ela é satisfeita quando
				\bea
					\hbar\det\rho &=& \l( \f{\hbar}{2} + z\r)\l( \f{\hbar}{2} - z\r) - (x-y)(x+y) \\
						&=& \f{\hbar^2}{4} - z^2 - x^2 + y^2 \\
						&=& 0 \\
						\f{\hbar^2}{4} &=& \braket{S_x}^2 - \lvert\braket{S_y}^2\rvert + \braket{S_z}^2
				\eea
				
			\end{multianswer}
		
			% Part 2
			\item Considere agora que o ensemble é misturado, de uma maneira genérica (note
			que não estamos dizendo que ele é o estado de máxima entropia!). Suponha agora
			que os três valores médios $\braket{S_x}, \braket{S_y}, \braket{S_z}$ são todos conhecidos. Construa explicitamente a matriz densidade $2\times 2$ que descreve o sistema.
			\begin{multianswer}[true]
				Sabemos que
				\be
					\braket{S_x} = \tr(\rho S_x), \quad \braket{S_y} = \tr(\rho S_y), \quad \braket{S_z} = \tr(\rho S_z)
				\ee
				Ademais
				\be
					\rho = \f{1}{2}\l( \mathbb{I} + \vec{b}\cdot\vec{\sigma} \r)
				\ee
				Como mostramos em sala, vale que
				\be
					\braket{\vec{S}} = \f{\hbar}{2} \vec{b}
				\ee
				Portanto,
				\be
					\rho = \f{1}{2} \l( \mathbb{I} + \f{2\braket{S_x}}{\hbar}\sigma_x + \f{2\braket{S_y}}{\hbar}\sigma_y + \f{2\braket{S_z}}{\hbar}\sigma_z \r)
				\ee
				Abrindo de maneira explícita:
				\be
					\rho = \f{1}{2} 
						\begin{pmatrix}
							1 & 0 \\
							0 & 1 \\ 
						\end{pmatrix}
						+ \f{1}{\hbar} 
						\begin{pmatrix}
							\braket{S_z} & \braket{S_x} - i\braket{S_y} \\
							\braket{S_x} +i\braket{S_y} & -\braket{S_z} \\
						\end{pmatrix}
				\ee
				
			\end{multianswer}
			
		\end{exercises}
	\end{exercise}
	
	
	\begin{exercise}
		Considere a seguinte matriz densidade de dois spins-$1/2$:
		\be
			\rho = \f{1}{8}\mathbb{I} + \f{1}{2} \ket{\Psi_-}\bra{\Psi_-}
 		\ee
		onde $\ket{\Psi_-}$ é o estado singleto (i.e. o estado de spin total igual a zero). Suponhamos que medimos um dos spins ao longo de um eixo $a$ e o outro ao longo de um eixo $b$, em que $\hat{a}\cdot\hat{b}=\cos\theta$. Qual é a probabilidade (como função de $\theta$) de encontramos $\hbar/2$ para ambos spins nestas medidas?
	\end{exercise}
	\begin{answer}
		Começamos notando que o estado de singleto é
		\be
			\ket{\Psi_-} = \f{1}{\sqrt{2}}\big( \ket{+-} - \ket{-+} \big)	
		\ee
		Em seguida, percebemos que calcular a probabilidades de $+\hbar/2$ em ambas as medidas é
		\be
			p_{++} = \tr( \ket{+}_a\ket{+}_b\bra{+}_a\bra{+}_b \rho)
		\ee
		Assim, sem perda de generalidade, vamos supor que $\hat{a}=\hat{z}$ e que $b$ mora no plano $x$-$z$. Então podemos escrever os kets em $b$ em função da base de $a$:
		\be
			\ket{+}_b = \cos\f{\theta}{2}\ket{+} + \sin\f{\theta}{2} \ket{-}, \quad \ket{-}_b = \sin\f{\theta}{2}\ket{+}  \cos{\theta}{2} \ket{-}
		\ee
		Portanto, o projetor se escreve:
		\bea
			P_{++} &=&  \ket{+}\l( \cos\f{\theta}{2}\ket{+} + \sin\f{\theta}{2} \ket{-}\r) \bra{+}\l(\cos\f{\theta}{2}\bra{+} + \sin\f{\theta}{2} \bra{-}\r) \\
				&=& \cos^2\f{\theta}{2}\ket{++}\bra{++} + \sin\f{\theta}{2}\cos\f{\theta}{2} \ket{++}\bra{+-} + \sin\f{\theta}{2}\cos\f{\theta}{2} \ket{+-}\bra{++} + \sin^2\f{\theta}{2} \ket{+-}\bra{+-} \\
				&=& \cos^2\f{\theta}{2}\ket{++}\bra{++} + \f{1}{2} \sin\theta \ket{++}\bra{+-} + \f{1}{2} \sin\theta\ket{+-}\bra{++} + \sin^2\f{\theta}{2} \ket{+-}\bra{+-}
		\eea
		Consequentemente, a probabilidade é
		\bea
			p_{++} &=& \tr{P_++}\rho) \\
				&=& \f{1}{8} \tr(P_{++}) + \f{1}{2} \tr(P_{++}\ket{\Psi_-}\bra{\Psi_-}) \\
				&=& \f{1}{8} \l( \cos^2\f{\theta}{2} + \sin^2\f{\theta}{2} \r) + \f{1}{2} \l( \f{1}{2} \l( \sin^2\f{\theta}{2} \r) \r) \\
				&=& \f{1}{8} + \f{1}{4}\sin^2\l(\f{\theta}{2}\r)
		\eea
	\end{answer}

	\begin{exercise}
		Suponha que um sistema de dois níveis tem estados $\ket{+}$ e $\ket{-}$, análogo ao spin-$1/2$. 
		\begin{exercises}
			\item Vamos considerar a chamada interação de beam splitter (BS). Suponha que o
			sistema é sujeito a um Hamiltoniano $H=\varepsilon\sigma_y$ por um tempo $\tau$ tal que $\varepsilon\tau/\hbar = \pi/4$. Mostre que se o sistema estava inicialmente em $\ket{+}$ ou $\ket{-}$, então ele é levado (a menos de uma fase global) para os estados:
			\be
				\ket{+} \to \f{\ket{+}+\ket{-}}{\sqrt{2}}, \quad\quad \ket{-} \to \f{\ket{+} - \ket{-}}{\sqrt{2}}
			\ee
			\begin{multianswer}
				Primeiramente, notamos que a evolução temporal é dada por
				\be
					U(t) = e^{-iHt/\hbar}
				\ee
				Além disso, sabemos que
				\bea
					\ket{+}_y &=& \f{\ket{+} + i\ket{-}}{\sqrt{2}} \\
					\ket{-}_y &=& \f{\ket{+} - i\ket{-}}{\sqrt{2}} 
				\eea
				Assim, somando e subtraindo essas equações, obtemos:
				\bea
					\ket{+} &=& \f{\ket{+}_y + \ket{-}_y}{\sqrt{2}} \\
					\ket{-} &=& -i \f{\ket{+}_y - \ket{-}_y}{\sqrt{2}} 
				\eea	
				onde lembramos que a ausência de índices implica que os vetores apontam na direção $\hat{z}$. Agora podemos fazer a projeção temporal:
				\bea
					e^{-iH\tau/\hbar}\ket{+} &=& e^{-i\pi\sigma_y/4} \l( \f{\ket{+}_y + \ket{-}_y}{\sqrt{2}}\r) \\
						&=& \f{1}{\sqrt{2}} \l( e^{-i\pi/4}\ket{+}_y + e^{i\pi/4}\ket{-}_y \r) \\
						&=& \f{1}{2} \l( \ket{+}_y - i\ket{+}_y + \ket{-}_y +i\ket{-}_y \r) \\
						&=& \f{1}{\sqrt{2}} \l( \f{\ket{+}_y + \ket{-}_y}{\sqrt{2}} -i \f{ \ket{+}_y - \ket{-}_y}{\sqrt{2}} \r) \\
						&=& \f{\ket{+} + \ket{-}}{\sqrt{2}} 
				\eea
				que é a evolução temporal desejada. Para o outro vetor, segue processo análogo:
				\bea
					e^{-iH\tau/\hbar}\ket{-} &=& -i e^{-i\pi\sigma_y/4} \l( \f{\ket{+}_y - \ket{-}_y}{\sqrt{2}}\r) \\	
						&=& -\f{i}{\sqrt{2}} \l( e^{-i\pi/4}\ket{+}_y - e^{i\pi/4}\ket{-}_y \r) \\
						&=& -\f{i}{2} \l( \ket{+}_y - i\ket{+}_y - \ket{-}_y - i\ket{-}_y \r) \\
						&=& \f{1}{2} \l( -i\ket{+}_y - \ket{+}_y + i\ket{-}_y - \ket{-}_y \r) \\
						&=& \f{1}{\sqrt{2}} \l( - \f{\ket{+}_y + \ket{-}_y}{\sqrt{2}} -i \f{\ket{+}_y - \ket{-}_y}{\sqrt{2}} \r) \\
						&=& \f{-\ket{+} + \ket{-}}{\sqrt{2}} 
				\eea
				* corrigir sinal
			\end{multianswer}
			
			% Part 2
			\item Agora, consideremos a chamada phase shift gate (PS). Suponha agora que o
			sistema é sujeito a um Hamiltoniano $H=\varepsilon\sigma_z$ por um tempo $\tau$ tal que $\varepsilon\tau/\hbar = \phi/2$. Mostre que isso leva os vetores (a menos de uma fase global) em
			\be
				\ket{+} \to \ket{+}, \quad\quad \ket{-} \to e^{i\phi}\ket{-}
			\ee
			\begin{multianswer}
				Como o hamiltoniano possui autoestados $\ket{+}, \ket{-}$, não precisamos fazer nenhuma projeção de spin, de modo que é imediato o resultado. Vejamos
				\bea
					e^{-iH\tau/\hbar}\ket{+} &=& e^{-i\sigma_z\phi/2} \ket{+} \\
						&=& e^{-i\phi/2} \ket{+}
				\eea
				Já o outro:
				\bea
					e^{-iH\tau/\hbar}\ket{-} &=& e^{-i\sigma_z\phi/2} \ket{-} \\
					&=& e^{i\phi/2} \ket{-} 
				\eea
				Como fazes globais não afetam medições, podemos multiplicar ambos os estados por $e^{i\phi/2}$ para obter os estados desejados
				\be
					\ket{+}, \quad\quad e^{i\phi}\ket{-}
				\ee
				a menos de alguma fase global. 
			\end{multianswer}
			
			% Part 3
			\item Considere um sistema inicialmente preparado em $\ket{\Psi_0}=\ket{+}$. Suponha agora que esse sistema evolui por um beam splitter, seguido de um phase shift, seguido de outro beam splitter, evoluindo até um estado $\ket{\Psi}$. Note que isto é possível em laboratório, usando exatamente um sistema de spin-$1/2$ e um campo magnético dependente do tempo. Calcule o estado final  $\ket{\Psi}$.
			\begin{multianswer}
				Vamos calcular diretamente passo-a-passo, começando do começo:
				\bea
					\ket{+} &\to_{BS}& \f{\ket{+} + \ket{-}}{\sqrt{2}} \\
						&\to_{PS}& \f{\ket{+} + e^{i\phi}\ket{-}}{\sqrt{2}} \\
						&\to_{BS}& \f{1}{\sqrt{2}} \l( \f{\ket{+} + \ket{-}}{\sqrt{2}}\r) + \f{e^{i\phi}}{\sqrt{2}} \l( \f{\ket{+} - \ket{-}}{\sqrt{2}} \r) \\
						&=& \f{1}{2} \l( \l(1 + e^{i\phi} \r)\ket{+} + \l(1 - e^{i\phi} \r) \ket{-} \r)
				\eea
				que é o estado final desejado.				
			\end{multianswer}
			
			% Part 4
			\item Mostre que a probabilidade $P_1$ de encontrar o sistema no estado $\ket{-}$ depois da evolução é dada por
			\be
				P_- = \l| \braket{- | \psi} \r|^2 = \sin^2(\phi/2)
			\ee
			Essa é a ideia principal por trás da interferometria de Ramsey: a probabilidade
			oscila dependendo da fase do phase shift.
			\begin{multianswer}[true]
				Ora, a probabilidade é trivial de se calcular nesse caso:
				\bea
					P_- &=& \l| 1-e^{i\phi} \r|^2 \\
						&=& \l| (1-\cos\phi) -i\sin\phi \r|^2 \\
						&=& (1-\cos\phi)^2 + \sin^2\phi \\
						&=& 1 - 2\cos\phi + \cos^2\phi + \sin^2\phi \\
						&=& 2 - 2\cos\phi \\
						&=& \sin^2(\phi/2)
				\eea
			\end{multianswer}
		\end{exercises}		
	\end{exercise}
	
\end{document}