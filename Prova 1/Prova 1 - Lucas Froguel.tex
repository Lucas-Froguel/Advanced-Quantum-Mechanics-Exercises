\documentclass[12pt]{article}
\usepackage{amsfonts}
\usepackage{amsthm}
\usepackage{amsmath}
\usepackage[mathscr]{euscript}
\usepackage{array}
\usepackage[thinlines]{easytable}
\usepackage{tikz}
\usepackage{pgfplots}
\usepackage[margin=2cm]{geometry}
\usepackage{graphicx}
\usepackage{xcolor}
\usepackage[utf8]{inputenc}
\usepackage[T1]{fontenc}
\usepackage[portuguese]{babel}
\usepackage{braket}
\usepackage{thmtools} 
\usepackage{hyperref}
\usepackage{environ}
\usepackage{enumitem}
\usepackage[backend=biber, style=numeric]{biblatex} %Imports biblatex package
\addbibresource{references.bib} %Import the bibliography file
\usetikzlibrary{angles,quotes}

\usepackage{lipsum} % Pode tirar esse :D

\graphicspath{.Images/}

\hypersetup{
	colorlinks=true,
	linkcolor=blue,
	filecolor=magenta,      
	urlcolor=cyan,
	pdftitle={Quântica Avançada - P1 - Lucas Froguel}
}


\def\be{\begin{equation}}
	\def\ee{\end{equation}}
\def\bee{\begin{equation*}}
	\def\eee{\end{equation*}}
\def\bea{\begin{eqnarray*}}
	\def\eea{\end{eqnarray*}}
\def\beaa{\begin{eqnarray}}
	\def\eeaa{\end{eqnarray}}

\def\f{\frac}
\def\del{\partial}

\def\R{\mathbb{R}}
\def\K{\mathbb{K}}
\def\C{\mathbb{C}}
\def\I{\mathbb{I}}
\def\Z{\mathbb{Z}}
\def\Q{\mathbb{Q}}
\def\N{\mathbb{N}}

\def\cd{\cdot}

\def\v#1{{\boldsymbol{#1}}}
\def\ve#1{\hat{\boldsymbol#1}}

\def\l{\left}
\def\r{\right}
\def\la{\l\langle}
\def\ra{\r\rangle}
\def\div{\nabla\cdot}
\def\curl{\nabla\times}
\def\grad{\nabla}
\def\lap{\nabla^2}

\def\s{\quad}
\def\ss{\qquad}
\def\infi{\infty}
\def\p{\partial}
\def\u{\cup}%union of two sets
\def\i{\cap}%intersection of two sets
\def\ds{\oplus}



\newtheorem{exercise}{Exercício}
\newtheorem{partinner}{Item}[exercise]

\newlist{exercises}{enumerate}{1}
\setlist[exercises]{wide = 0pt, listparindent=\parindent,labelsep = 0pt,leftmargin =\labelwidth}
\setlist[exercises, 1]{label =\itshape \bfseries Item~\arabic*.\; }

\newenvironment{answer}{\noindent\textbf{\textit{Resposta.}} \normalfont }{\par\noindent\rule{\textwidth}{0.4pt}}
\NewEnviron{multianswer}[1][false]{%
	\ifthenelse{\equal{#1}{true}}%
	{\def\rulewidth{\textwidth}}%
	{\def\rulewidth{0.7\textwidth}}%
	\\ \noindent\textbf{\textit{Resposta.}} \normalfont%
	\BODY%
	\par\noindent\rule{\rulewidth}{0.1pt}%
}


\newcommand\norm[1]{\left\lVert#1\right\rVert}

\DeclareMathOperator{\atan}{atan}
\DeclareMathOperator{\cotan}{cotan}
\DeclareMathOperator{\acos}{acos}
\DeclareMathOperator{\sech}{sech}
\DeclareMathOperator{\csch}{csch}
\DeclareMathOperator{\asinh}{asinh}
\DeclareMathOperator{\atanh}{atanh}
\DeclareMathOperator{\acoth}{acoth}
\DeclareMathOperator{\acosh}{acosh}
\DeclareMathOperator{\acsch}{acsch}
\DeclareMathOperator{\asech}{asech}
\DeclareMathOperator{\D}{D}
\DeclareMathOperator{\tr}{tr}
\DeclareMathOperator{\Res}{Res}

\title{Mecânica Quântica Avançada \\ Prova 1}
\author{Lucas Froguel \\ IFT}
\date{}

\begin{document}
	\maketitle
	\listoftheorems[title={Prova 1}]

	\begin{exercise}
		Considere uma molécula constituída de três átomos idênticos nos
		sítios (vértices) de um triângulo equilátero, conforme mostrado na figura. Vamos considerar que o íon desta molécula é formado adicionando-se um único elétron a ela; como vimos em aula, deve haver um efeito de delocalização, e o elétron deve se acomodar em um estado distribuído ao redor dos sítios. Suponha que o elemento de matriz da Hamiltoniana para o elétron em dois sítios adjacentes $i, j$ é $\braket{i|H|j} = -a, a>0$, para $i\not=j$; por simetria, os elementos diagonais são todos	iguais, $\braket{i|H|i}=E_0$.
		\begin{exercises}
			\item Escreva a Hamiltoniana na forma matricial e calcule os níveis de energia.
			\begin{multianswer}
				A hamiltoniana é
				\be
					H = 
					\begin{pmatrix}
						E_0 & -a & -a \\
						-a & E_0 & -a \\
						-a & -a & E_0 \\
					\end{pmatrix}
				\ee
				Diagonalizando a matriz (processo feito em aula), achamos que as autoenergias são
				\be
					E_s = E_0 -2a\cos\l( \f{2\pi s}{3}\r)
				\ee
				Assim, nossas energias são
				\be
					E_0 = E_0 - 2a, \quad E_1 = E_0 -2a\cos\l(\f{2\pi}{3}\r), \quad E_2 = E_0 -2a\cos\l(\f{4\pi}{3}\r)
				\ee
			\end{multianswer}
			
			% Part 2
			\item Suponha que um campo elétrico na direção $z$ é aplicado, de modo que a energia potencial para o elétron na posição rotulada por “1” diminui por uma parcela $b$
			$(b>0)$. Calcule os níveis de energia e os autoestados de energia.
			\begin{multianswer}
				Agora quebramos a simetria que fazia com que os elementos das diagonais fossem $E_0$, pois podemos distinguir o estado 1 dos estados 2 e 3. Logo, a hamiltoniana modificada é 
				\be
				H = 
				\begin{pmatrix}
					E_0 - b & -a & -a \\
					-a & E_0 + b & -a \\
					-a & -a & E_0 + b \\
				\end{pmatrix}
				\ee
			\end{multianswer}
			
		\end{exercises}
	\end{exercise}


\end{document}